%%%%%%%%%%%%%%%%%%%%%%%%%%%%%%%%%%%%%%%%%
% Medium Length Professional CV
% LaTeX Template
% Version 3.0 (December 17, 2022)
%
% This template originates from:
% https://www.LaTeXTemplates.com
%
% Author:
% Vel (vel@latextemplates.com)
%
% Original author:
% Trey Hunner (http://www.treyhunner.com/)
%
% License:
% CC BY-NC-SA 4.0 (https://creativecommons.org/licenses/by-nc-sa/4.0/)
%
%%%%%%%%%%%%%%%%%%%%%%%%%%%%%%%%%%%%%%%%%

%----------------------------------------------------------------------------------------
%	PACKAGES AND OTHER DOCUMENT CONFIGURATIONS
%----------------------------------------------------------------------------------------

\documentclass[
	%a4paper, % Uncomment for A4 paper size (default is US letter)
	10pt, % Default font size, can use 10pt, 11pt or 12pt
]{resume} % Use the resume class

\usepackage{utopia} % Use the EB Garamond font
\usepackage[symbol, flushmargin]{footmisc}
\usepackage{fancyhdr}
\usepackage{lastpage}
\usepackage{multicol}

\usepackage{hyperref}
\hypersetup{
    colorlinks=true,
    linkcolor=black,
    filecolor=magenta,      
    urlcolor=blue
    }

\pagestyle{fancy}
\cfoot{Page \thepage\ of \pageref{LastPage}}
\renewcommand{\headrulewidth}{0pt}

%------------------------------------------------

\name{Kristian Grünberg Olesen Larsen} % Your name to appear at the top

% You can use the \address command up to 3 times for 3 different addresses or pieces of contact information
% Any new lines (\\) you use in the \address commands will be converted to symbols, so each address will appear as a single line.
\address{kristianuruplarsen@gmail.com \\ (+45) 22 93 02 04 } % Contact information
\address{Valborg Allé 19, 5.tv., Copenhagen, Denmark} % Main address

%----------------------------------------------------------------------------------------

\begin{document}

%----------------------------------------------------------------------------------------
%	EDUCATION SECTION
%----------------------------------------------------------------------------------------

\begin{rSection}{Education}
    \textbf{University of Copenhagen}, \textit{PhD in Economics} \hfill \textit{2019-2023} \\
    Supervisors: Søren Leth-Petersen \& Claus Thustrup Kreiner \\
    Thesis: {Gender Inequality, Labor Supply and New Structural Methods}

    \textbf{University of Copenhagen}, \textit{MSc in Economics (cand.polit.), GPA: 10.9/12} \hfill \textit{2018-2021\footnote{I enrolled in the PhD program on a ``4+4 contract'', meaning i was enrolled simultaneously in the MSc and PhD for the first half of my PhD studies.}} \\
    Thesis: {Child Penalties in Labor Returns? Evidence from Linked Survey and Administrative Data}

    \textbf{University of Copenhagen}, \textit{BSc in Economics (ba.polit.), GPA: 10.2/12} \hfill \textit{2014-2017} \\
    Thesis: {Does Displacement Effects Persist to the Next Generation? Evidence from Danish Childrens Performance in Primary School Exams}
\end{rSection}

%----------------------------------------------------------------------------------------
%	WORK EXPERIENCE SECTION
%----------------------------------------------------------------------------------------

\begin{rSection}{Experience}
    \begin{rSubsection}{CEBI, Department of Economics, University of Copenhagen}{2023-Present}{Postdoctoral Researcher}{}
        \item[] Independent empirical and theoretical research and publication of research projects in scientific journals.
    \end{rSubsection}

    \begin{rSubsection}{CEBI, Department of Economics, University of Copenhagen}{2019-2023}{PhD Stipend}{}
        \item[] Independent empirical and computational research, research project design, university teaching, research dissemination. I worked extensively with Professor Henrik Kleven from Princeton University through all four years on research in optimal taxation, wrote a solo-author paper in labor economics and conducted computational research leveraging state of the art methods from reinforcement learning. I managed day-to-day tasks of 4-8 student research assistants.
    \end{rSubsection}

    \begin{rSubsection}{CEBI, Department of Economics, University of Copenhagen}{2017-2019}{Research Assistant}{}
        \item[] Empirical research assistance using Stata, SAS, and R to analyze large datasets. I primarily worked on health economic projects that have resulted in publications in PNAS and ReStat.
    \end{rSubsection}

    %    \textbf{University of Copenhagen (Department of Economics)}\hfill \textit{2023 - Present} \\
    %    Postdoctoral Researcher
    %
    %    \textbf{University of Copenhagen (Department of Economics)}\hfill \textit{2019 - 2023} \\
    %    PhD Stipend
    %
    %    \textbf{University of Copenhagen (Department of Economics)}\hfill \textit{2017 - 2019} \\
    %    Student Research Assistant at Center for Economic Behavior and Inequality (CEBI).

\end{rSection}


\begin{rSection}{Teaching Experience}

    \textbf{Seminar: Applications of Machine Learning in Economics (MSc)}\hfill \textit{Spring 2021} \\
    Course coordinator and supervisor, economics MSc program, Department of Economics.

    \textbf{Social Data Science: Econometrics and Machine Learning}\hfill \textit{Spring 2020} \\
    Course development \& teaching assistant, economics MSc/PhD program, SODAS.

    \textbf{Social Data Science}\hfill \textit{August 2019} \\
    Course management and teaching assistant, economics MSc/BSc program, SODAS.

    \textbf{Topics in Social Data Science}\hfill \textit{Fall 2018} \\
    Teaching assistant, economics PhD/MSc/BSc program, SODAS.

    \textbf{Social Data Science}\hfill \textit{August 2018} \\
    Teaching assistant, economics MSc/BSc program, SODAS.

\end{rSection}

\newpage
%----------------------------------------------------------------------------------------
%	Research
%----------------------------------------------------------------------------------------

\begin{rSection}{Research}

    \textbf{Micro vs Macro Labor Supply Elasticities: The Role of Dynamic Returns to Effort}\hfill \textit{[\href{https://www.nber.org/papers/w31549}{Working Paper}]} \\
    with Henrik Kleven, Claus Thustrup Kreiner \& Jakob Egholt Søgaard \\
    \textit{R\&R American Economic Review}

    \textbf{Couples and Gender Inequality}\hfill \textit{[\href{https://ssrn.com/abstract=4697847}{Working Paper}]} \\
    Solo Paper

    \textbf{Using Reinforcement Learning for Solving Dynamic Discrete-time Problems in Economics} \\
    w. Joachim Kahr Rasmussen, \textit{working paper upon request}
\end{rSection}

%----------------------------------------------------------------------------------------
%	Services
%----------------------------------------------------------------------------------------

\begin{rSection}{Services \& Voluntary Work}

    \textbf{The Danish Mountain and Climbing Club (DBKK)}, \textit{Board Member} \hfill \textit{2024} \\
    \textbf{Studenterrådet ved Københavns Universitet}, \textit{Board Member} \hfill \textit{2016-2018} \\
    \textbf{The UCPH Academic Board on Education Strategy (KUUR)}, \textit{Student Representative} \hfill \textit{2017} \\
    \textbf{The Academic Council of the Faculty of Social Sciences}, \textit{Student Representative} \hfill \textit{2017} \\

\end{rSection}

%----------------------------------------------------------------------------------------
%	TECHNICAL STRENGTHS SECTION
%----------------------------------------------------------------------------------------

\begin{rSection}{Other}

    \begin{tabular}{@{} >{\bfseries}l @{\hspace{6ex}} l @{}}
        Programming      & Daily use of Python, R and Stata. Experienced with SAS and SQL. \\
        Spoken Languages & Fluent Danish and English.                                      \\
    \end{tabular}

\end{rSection}


% \begin{rSection}{References}
%     Claus Thustrup Kreiner - \textit{Professor, CEBI}    \\
%     Relation: Supervised me during my PhD studies. \\
%     Contacts: ctk@econ.ku.dk

%     Andreas Bjerre-Nielsen  - \textit{Associate Professor, SODAS}    \\
%     Relation: Ran most of the data science courses I co-taught. \\
%     Contacts: abn@econ.ku.dk

% \end{rSection}

%----------------------------------------------------------------------------------------
%	EXAMPLE SECTION
%----------------------------------------------------------------------------------------

%\begin{rSection}{Section Name}

%Section content\ldots

%\end{rSection}

%----------------------------------------------------------------------------------------

\end{document}
