%%%%%%%%%%%%%%%%%%%%%%%%%%%%%%%%%%%%%%%%%
% Medium Length Professional CV
% LaTeX Template
% Version 3.0 (December 17, 2022)
%
% This template originates from:
% https://www.LaTeXTemplates.com
%
% Author:
% Vel (vel@latextemplates.com)
%
% Original author:
% Trey Hunner (http://www.treyhunner.com/)
%
% License:
% CC BY-NC-SA 4.0 (https://creativecommons.org/licenses/by-nc-sa/4.0/)
%
%%%%%%%%%%%%%%%%%%%%%%%%%%%%%%%%%%%%%%%%%

%----------------------------------------------------------------------------------------
%	PACKAGES AND OTHER DOCUMENT CONFIGURATIONS
%----------------------------------------------------------------------------------------

\documentclass[
	%a4paper, % Uncomment for A4 paper size (default is US letter)
	11pt, % Default font size, can use 10pt, 11pt or 12pt
]{resume} % Use the resume class
\usepackage[symbol, flushmargin]{footmisc}
\usepackage{utopia} % Use the EB Garamond font
\usepackage{hyperref}
\hypersetup{
    colorlinks=true,
    linkcolor=black,
    filecolor=magenta,      
    urlcolor=blue
    }
%------------------------------------------------

\name{Kristian Urup Olesen Larsen} % Your name to appear at the top

% You can use the \address command up to 3 times for 3 different addresses or pieces of contact information
% Any new lines (\\) you use in the \address commands will be converted to symbols, so each address will appear as a single line.

\address{(+45) 22 93 02 04 \\ kristianuruplarsen@gmail.com} % Contact information
\address{Valborg Allé 19, 5.tv. \\ København, Danmark} % Main address

%----------------------------------------------------------------------------------------

\begin{document}

%----------------------------------------------------------------------------------------
%	EDUCATION SECTION
%----------------------------------------------------------------------------------------

\begin{rSection}{Uddannelse}
    \textbf{PhD Polit.}, \textit{Københavns Universitet} \hfill \textit{2019-2023} \\
    Vejledere: Søren Leth-Petersen \& Claus Thustrup Kreiner \\
    Afhandling: \textit{Gender Inequality, Labor Supply and New Structural Methods}

    \textbf{Cand. Polit.}, \textit{Københavns Universitet (Gns. 10.9/12)} \hfill \textit{2018-2021\footnote{Jeg var indskrevet på PhD-skolen på en ``4+4 kontrakt'', hvorfor jeg var både kandidat- og PhD-studerende indtil halvejs igennem PhD-forløbet.}} \\
    Speciale: \textit{Child Penalties in Labor Returns? Evidence from Linked Survey and Administrative Data}

    \textbf{Ba. Polit.} \textit{Københavns Universitet (Gns. 10.2/12)} \hfill \textit{2014-2017} \\
    Bacheloropgave: \textit{Does Displacement Effects Persist to the Next Generation? Evidence from Danish Childrens Performance in Primary School Exams}
\end{rSection}

%----------------------------------------------------------------------------------------
%	WORK EXPERIENCE SECTION
%----------------------------------------------------------------------------------------

\begin{rSection}{Relevante Stillinger}

    \textbf{Københavns Universitet, Økonomisk Institut}\hfill \textit{2023 - Nu} \\
    Postdoctoral Researcher

    \textbf{Københavns Universitet, Økonomisk Institut}\hfill \textit{2019 - 2023} \\
    PhD Stipendiat

    \textbf{Københavns Universitet, Økonomisk Institut}\hfill \textit{2017 - 2019} \\
    Forskningsassistent ved \textit{Center for Economic Behavior and Inequality (CEBI)}.

\end{rSection}

%----------------------------------------------------------------------------------------
%	Research
%----------------------------------------------------------------------------------------

\begin{rSection}{Forskning}

    \textbf{Micro vs Macro Labor Supply Elasticities: The Role of Dynamic Returns to Effort}\hfill \textit{[\href{https://www.nber.org/papers/w31549}{Working Paper}] R\&R AER} \\
    med Henrik Kleven, Claus Thustrup Kreiner \& Jakob Egholt Søgaard

    \textbf{Couples and Gender Inequality}\hfill \textit{[\href{https://ssrn.com/abstract=4697847}{Working Paper}]} \\
    Solo papir

    \textbf{Using Reinforcement Learning for Solving Dynamic Discrete-time Problems in Economics} \\
    med Joachim Kahr Rasmussen, \textit{working paper kan fremsendes på forespørgsel}
\end{rSection}


\begin{rSection}{Undervisning}

    \textbf{Seminar: Applications of Machine Learning in Economics (MSc)}\hfill \textit{Forår 2021} \\
    Kursuskoordinator og vejleder (med J.K. Rasmussen), valgfag på Ka. niveau

    \textbf{Social Data Science: Econometrics and Machine Learning}\hfill \textit{Forår 2020} \\
    Kursusudvikling \& undervisningsassistent, valgfag på Ka./PhD. niveau

    \textbf{Social Data Science}\hfill \textit{August 2019} \\
    Undervisningsassistent, valgfag på Ba./Ka. niveau

    \textbf{Topics in Social Data Science}\hfill \textit{Efterår 2018} \\
    Undervisningsassistent, valgfag på Ba./Ka./PhD. niveau

    \textbf{Social Data Science}\hfill \textit{August 2018} \\
    Undervisningsassistent, valgfag på Ba./Ka. niveau
\end{rSection}

\newpage
%----------------------------------------------------------------------------------------
%	Services
%----------------------------------------------------------------------------------------

\begin{rSection}{Frivillighed}

    \textbf{Dansk Bjerg og Klatreklub (DBKK)}, \textit{Bestyrelsesmedlem} \hfill \textit{2024} \\
    \textbf{Studenterrådet ved Københavns Universitet}, \textit{Bestyrelsesmedlem} \hfill \textit{2016-2018} \\
    \textbf{Københavns Universitets Uddannelsesstrategiske Råd (KUUR)}, \textit{Studenterrepræsentant} \hfill \textit{2017} \\
    \textbf{Akademisk Råd SAMF}, \textit{Studenterrepræsentant} \hfill \textit{2017} \\

\end{rSection}

%----------------------------------------------------------------------------------------
%	TECHNICAL STRENGTHS SECTION
%----------------------------------------------------------------------------------------

\begin{rSection}{Diverse}

    \begin{tabular}{@{} >{\bfseries}l @{\hspace{6ex}} l @{}}
        Programmeringssprog & Python, R, STATA, SAS     \\
        Talte sprog         & Flydende dansk og engelsk \\
    \end{tabular}

\end{rSection}

%----------------------------------------------------------------------------------------
%	EXAMPLE SECTION
%----------------------------------------------------------------------------------------

%\begin{rSection}{Section Name}

%Section content\ldots

%\end{rSection}

%----------------------------------------------------------------------------------------

\end{document}
